During online purchases customers often send back items they order. These returns are costly and due to the high competition in online retail it is not possible or highly inadvisible to pass on the costs of return shipping to the customer. Therefore, accurate predictions of product returns could allow online retailers to impede problematic transactions, for example by restricting payment options or by displaying a warning message and thus cut down on costs incurred by shipping. 

The aim of this analysis is to provide a suitable predictive machine to this problem. To this effect, a german online retail platform has provided data about the online purchasing behaviour of its customers along with the information if the item has been returned or not. This paper applies multiple machine learning approaches to this data and discusses the results. The scripts for this report are written in the \texttt{Python} programming language and are available online at \href{https://github.com/thsis/bads19}{https://github.com/thsis/bads19} \cite{python, scipy, pyplot, seaborn}.

Section \ref{eda} contains a description of the data and the exploratory results. Section \ref{preparation} describes the actions taken in order to clean the data and a brief overview of the efforts taken during feature engineering. Section \ref{tuning} specifies the different algorithms that were deployed and section \ref{evaluation} includes a succinct discussion of the results. Section \ref{special} describes how to minimize the costs of misclassifying an item directly by using a custom built and modified Genetic Algorithm. Finally, section \ref{conclusion} contains concluding remarks. 